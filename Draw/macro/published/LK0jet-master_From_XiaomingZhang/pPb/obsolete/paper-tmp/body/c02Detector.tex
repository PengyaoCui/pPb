%!TEX root = ../AliPubV0JetspPb.tex

\section{Detector setup}

The ALICE apparatus~\cite{Aamodt:2008zz} consists of a central barrel detector covering the pseudo-rapidity interval $|\eta|<0.9$, a forward muon spectrometer covering the pseudo-rapidity interval $-4.0<\hlab{\eta}<-2.5$, and a set of detectors at forward and backward rapidities used for triggering and event characterization. 

The central barrel detectors are located inside a large solenoidal magnet, which provides a magnetic field of 0.5 T along the beam direction ($z$ axis in the ALICE reference frame). 
Tracking and particle identification are performed using the information provided by the Inner Tracking System (ITS) \cite{Aamodt:2010aa}, the Time Projection Chamber (TPC) \cite{Alme:2010ke} and the Time Of Flight (TOF) \cite{Akindinov:2013tea} detectors, that have full azimuthal coverage in the pseudo-rapidity interval $|\hlab|<0.9$. 

The detector closest to the beam axis is the ITS, which is composed of six cylindrical layers of silicon detectors, with radial distances from the beam axis ranging from 3.9~cm to 43.0~cm. 
The two innermost layers, with average radii of 3.9~cm and 7.6~cm, are equipped with Silicon Pixel Detectors (SPD). 
The two SPD layers, covering the pseudo-rapidity ranges of $|\hlab|< 2.0$ and $|\hlab|< 1.4$ respectively, have 1200 SPD readout chips.  
The two intermediate layers are made of Silicon Drift Detectors (SDD), while Silicon Strip Detectors (SSD) equip the two outermost layers. 
The high spatial resolution of the silicon sensors, together with the low material budget (on average 7.7\% of a radiation length for tracks crossing the ITS perpendicularly to the detector surfaces, i.e.\ $\hlab=0$) and the small distance of the innermost layer from the beam vacuum tube, allow for the measurement of the track impact parameter in the transverse plane ($d_0$), i.e.\ the distance of closest approach of the track to the primary vertex in the plane transverse to the beam direction, with a resolution better than 75~$\mu$m for transverse momenta $\pt>1~\gevc$~\cite{Aamodt:2010aa}.
The SPD provides also a measurement of the multiplicity of charged particles produced in the collision based on track segments (tracklets) built by associating pairs of hits in the two SPD layers.

At larger radii ($85<r<247~\cm$), a 510 cm long cylindrical TPC provides track reconstruction with up to 159 three-dimensional space points per track, as well as particle identification via the measurement of the specific energy deposit $\dedx$ in the gas.
The charged particle identification capability of the TPC is supplemented by the TOF, which is equipped with Multi-gap Resistive Plate Chambers  (MRPCs) located at radial distances between 377 and 399 cm from the beam axis. The overall TOF resolution including the uncertainty on the time at which the collision took place, and the tracking and momentum resolution was about 160~ps for the data-taking period considered in these analyses. 

The V0 detector~\cite{Abbas:2013taa}, used for triggering and for estimating the multiplicity of charged particles in the forward rapidity region, consists of two arrays of 32 scintillator tiles each, placed around the beam vacuum tube on either side of the interaction region at $z =-90$ cm and $z=+340$ cm. The two arrays cover the pseudo-rapidity intervals $-3.7 < \hlab < -1.7$ (VZERO-C) and $2.8 < \hlab < 5.1$ (VZERO-A), respectively. In addition two Zero Degree Calorimeters (ZDCs) located at $+112.5$ m (\ZNA) and $-112.5$ m (\ZNC) from the interaction point were used for beam background rejection and an alternative estimator of the event activity.

%%Only events with interaction vertex reconstructed from tracks with a coordinate $|z|<10$~cm along the beam line were used for the analysis.

%%\ask{this and the next section are fixed in the next version}
