\subsection{Jet reconstruction}
\label{sec:JetReco}

In this Letter, the hard scatterings are tagged by the charged particle jets.

Charged particles were reconstructed as tracks in the ITS and the TPC and required to point close to the primary vertex measured by the SPD.
Due to the SPD inefficient, azimuthal distribution of these high quality tracks is not fully uniform and causes the biases of the jet reconstruction algorithm.
This was compensated by tracks without reconstructed track points in the SPD.
For those tracks, the primary vertex is used as an additional constraint in the track fitting to improve the momentum resolution.
The combined track sample is called as the hybrid tracks.
The fraction of additional tracks (without SPD track points) is $\sim 4.3\%$ of the hybrid tracks.
For each track, the Distance-of-Closest Approach (DCA) to primary vertex is required $<2.4$~cm in transverse plane and $<3.1$~cm along the beam direction.
With this criterion, the tracks generated in detector material and from the weak decays are removed, the residual contamination is $\sim 10\%$ in $\pT<1~\GeVc$.
The overall efficiency for charged particle detection is $70\%$ at $\pT=0.15~\GeVc$ and increases to $85\%$ at $\pT=1~\GeVc$ and above.
In present analysis, the tracks in $\pT>0.15~\GeVc$ and $\abs{\hlab}<0.9$ are used in jet reconstruction. \ask{-- The description of the hybrid tracks is too long, can this be simplified?}

The jets have been reconstructed using the anti-$\kT$ algorithm~\cite{Cacciari:2008gp} from the FastJet package~\cite{Cacciari:2011ma,Cacciari:2005hq} with resolution parameters of $R=0.2$, $0.3$ and $0.4$. In the further analysis, the jets were considered if the jet-axis in the acceptance window $\abs{\hlab}<0.35$, which fully constrains the jets within the acceptances of both charged particle tracks ($\abs{\hlab}<0.9$) and $\Vzero$s ($\abs{\hlab}<0.75$, see section~\ref{sec:V0Reco} for the details) with all different used resolution parameters.
The analysis technique was described in detail in~\cite{Adam:2015hoa,Abelev:2013kqa}, hereafter is a brief review of the most relevant points.

In general, the background originating from the underlying event and/or pile-up (particles not associated to the hard scattering) impacts on the reconstructed jet energy.
The correction of the jet transverse momentum ($\pT[jet]^{\rm ch}$) scale accounting for the transverse background energy density, $\rho^{\rm ch}$, which can be estimated on event-by-event basis using the median of all jet candidate clusters $\pT$ reconstructed with the $\kT$ algorithm per unit area~\cite{Cacciari:2007fd,Abelev:2013kqa},
\begin{equation}
\pT[jet]^{\rm ch}=\pT[jet]^{\rm ch,raw}-\rho^{\rm ch}A_{\rm jet}^{\rm ch},
\end{equation}
where,  $A_{\rm jet}^{\rm ch}$  and $\pT[jet]^{\rm ch,raw}$ are the measured area~\cite{Cacciari:2008gn} and transverse momentum for the charged particle jets.
To further correct the fluctuations, caused by the more sparse environment of p--Pb events, of background density, an approach described in~\cite{Chatrchyan:2012tt} was employed.
The resulting mean and various of background density is $\avg{\rho^{\rm ch}}=1.02~\GeVc\times$rad$^{-1}$ and $\sigma(\rho^{\rm ch})=0.91~\GeVc\times$rad$^{-1}$ for all events and, $\avg{\rho^{\rm ch}}=2.2~\GeVc\times$rad$^{-1}$ and $\sigma(\rho^{\rm ch})=1.47~\GeVc\times$rad$^{-1}$ for events containing jet within $\pT[jet]^{\rm ch,raw}>20~\GeVc$~\cite{Adam:2015hoa}.
