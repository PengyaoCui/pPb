The production of $\Lambda$ baryons and $\Kshort$ mesons has been measured separately in hard scattering region and the underlying event in p--Pb collisions at $\sNN=5.02$~TeV at the LHC.
The hard scatterings are selected on an event-by-event basis by jets reconstructed using charged particles with anti-$\kT$ jet finder and resolution parameter $R=0.4$ (or $R=0.2$) (\ask{various $R$ vales were used in this analysis, since we will show the results of $\Lambda$-to-$\Kshort$ ratio vs. $R(\Vzero,{\rm jet})$}), and $\pT[jet]$ of $10$ (or $20)~\GeVc$.

The ratio of inclusive differential yields of $\Lambda$ and $\Kshort$ at intermediate transverse momentum ($\pT$) is much larger in the systems such as Pb--Pb and p--Pb collisions as compared to pp collisions.
The increased ratio in Pb--Pb has been attributed to collective effects in those collisions.
Recent studies have revealed qualitatively similar effects in high-multiplicity p--Pb collisions.
\ask{-- This paragraph needs to be simplified (?) This is more like the motivation -- in abstract we may concentrate more on the discussion of the results in this analysis (?)}

In this letter, we report that in p--Pb collisions the ratio of $\Lambda$ to $\Kshort$ associated to jets is consistent with pp expectation given by \textsc{Pythia} event generator.
Whereas, the ratio in the underlying event shows a prominent maximum (similar to the inclusive distribution) at the intermediate $\pT$ of $2$--$5~\GeVc$.
Moreover the yields in jets do not change with the event multiplicity, while the large baryon/meson ratio evolves and it is larger in events with the highest multiplicity as compared to minimum bias collisions.
\ask{-- Should we mention the multipliticy dependence? Since we will only the results in MB events in the main content of the paper.}