Table \ref{tab:v0cuts} presents the topological cuts applied on the candidate tracks of the decays daughters. 

\ask{See how much details is needed - since all follows from \cite{Abelev:2013haa}.}

The selection criteria used to define $\Vzero$ candidates are listed in table~\ref{tab:v0cuts} (see \cite{Aamodt:2011zza} for details).

The $\Vzero$ decay-product tracks are selected in the acceptance window $\abs{\hlab}<0.8$, only the candidates reconstructed in $\abs{\hlab}<0.75$ are retained to keep $>50\%$ of acceptance and reconstruction efficiency at plateau around $\hlab=0$ \ask{(is it clear or is it needed?)}.
A five-standard-deviation particle-identification cut was applied on the difference between $\dedx$ in the TPC and that defined by parameterized Bethe-Bloch curve for the $\Vzero$ decay-product tracks.
In addition, by using proper lifetime selection coupled to a momentum independent cosine of pointing angle ($\cos\theta_{\rm pointing}$) selection of $\cos\theta_{\rm pointing}>0.995$ majority of secondary $\lda$ ($\alda$) generated in detector material are removed.
The residual contamination entering the selections was $<1\%$ and was neglected~\cite{Abelev:2013xaa}.
The yield of $\Vzero$ signal is extracted from the invariant mass, $M_{\rm inv}$, distribution of identified $\Vzero$ candidates subtracting the combinatory background from the peak region with the bin counting method.
The background was determined by fitting first order polynomials to sideband regions.
The signal region and sidebands are defined on the basis of the $\pT$-dependent mass resolution as the windows in $\abs{M_{\rm inv}-M_{0}}<6\sigma$ and $6\sigma<\abs{M_{\rm inv}-M_{0}}<12\sigma$, respectively, where $M_{0}$ and $\sigma$ are the mean and width of invariant mass distribution extracted by the Gaussian fit.

\begin{table}[t]
  \centering
  \begin{tabular*}{\linewidth}{@{\extracolsep{\fill}}lr}
    \hline
    &\\[-0.7em]
    Selection variable & Cut value \\[0.3em]
    \hline
    &\\[-0.7em]
    2D decay radius & $>0.50$~cm \\[0.3em]
    Daughter track DCA to primary vertex & $>0.06~$cm \\[0.3em]
    DCA between daughter tracks & $<1.0~\sigma$ \\[0.3em]
    Cosine of pointing angle (\kzero) & $>0.97$ \\[0.3em]
    & ($<1\%$ signal loss) \\[0.3em]
    Cosine of pointing angle (\lmb\ and \almb) & $>0.995$ \\[0.3em]
    & ($<1\%$ signal loss) \\[0.3em]
    Proper lifetime (\kzero) & $<20$~cm \\[0.3em]
    Proper lifetime (\lmb\ and \almb) & $<30$~cm \\[0.3em]   
    \kzero\ mass rejection window (\lmb\ and \almb) & $\pm 10$~\mevc\ \\[0.3em]
    \lmb\ and \almb\ mass rejection window (\kzero) & $\pm 5$~\mevc\ \\[0.3em]
    \hline
  \end{tabular*}
  \caption{\vzero\ topological selection cuts (DCA: distance-of-closest approach). \ask{check if the cos theta \pt\ dependent(?)}}
  \label{tab:v0cuts}
\end{table}
