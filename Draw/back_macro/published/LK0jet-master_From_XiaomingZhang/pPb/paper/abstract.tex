%!TEX root = ../AliPubV0JetspPb.tex
The production of \lda\ baryons and \ks\ mesons has been measured separately in hard scattering region and the underlying event in \pPb\ collisions at \sqrtsnn{5.02} at the LHC.
The hard scatterings are selected on an event-by-event basis by jets reconstructed using charged particles with \akT\ jet finder. The production of the particles is reported depending on the angular distance \dr\ from the jet axis for jets with $\ptjet > 10$~\gevc and $\ptjet > 20$~\gevc.

%%The ratio of inclusive differential yields of \lda\ and \ks\ at intermediate transverse momentum (\pt) is much larger in the systems such as \PbPb\ and \pPb\ collisions as compared to \pp\ collisions. The increased ratio in \PbPb\ has been attributed to collective effects in those collisions. Recent studies have revealed qualitatively similar effects in high-multiplicity \pPb\ collisions.

For small \dr\ ($\dr\ < 1.0$) the ratio of \lda\ to \ks\ associated to jets is found consistent with the expectation of jets fragmenting in vacuum given by \pythia\ event generator. Whereas, the ratio for large \dr\ (i.e. the underlying event) shows a maximum (similar to the inclusive distribution in \pPb\ collisions) at the intermediate \pt\ of 2-5 \gevc. 

%Moreover the yields in jets do not change with the event multiplicity, while the large baryon/meson ratio evolves and it is larger in events with the highest multiplicity as compared to minimum bias collisions.


