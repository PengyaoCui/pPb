%!TEX root = ../AliPubV0JetspPb.tex

\section{Data sample}

The data used for this analysis was taken during the LHC p--Pb run at \sqrtsnn{5.02} in the beginning of $2013$.
Since the $2$-in-$1$ magnet design of the LHC~\cite{Evans:2008zzb}, the energy of the two beams are not independent and their ratio is fixed to equal to the ratio of the charge/mass ratios of each beam.
Comparing to the laboratory frame, the nucleon-nucleon center-of-mass system (cms), therefore, was shifting with a rapidity of $\Delta y_{\rm NN}=0.465$ in the direction of the proton beam.
The number of colliding bunches was varied from $8$ to $288$. The total number of proton and Pb ion bunch intensities ranged from $0.2\times 10^{12}$ to $6.5\times 10^{12}$ and from $0.1\times 10^{12}$ to $4.4\times 10^{12}$, respectively.
The luminosity at the ALICE interaction point for the data used in this analysis was up to $5\times 10^{27}$~cm$^{-2}$s$^{-1}$ resulting a $10$~kHZ hadronic interaction rate. The r.m.s of the interaction region is $6.3$~cm along the beam direction and $60~\mu{\rm m}$ in the direction to the transverse to the beam. The used data was collected for the beam configuration, in which the Pb beam circulated in the ``counter-clockwise'' direction, corresponding to travel from ALICE C to A side or positive rapidity.
%%%%%%%%%%%%%%%%%%%%%%%%%%%%%%%%%%%%%%%%%%%%%%%%%%%%%%%%%%%%%%%%%%%%%%%%%%%%%%%

\subsection{Event selection}

The minimum-bias trigger signal was provided by the VZERO counters, the VZERO-A in Pb beam direction and the VZERO-C in proton beam direction. The signal amplitude and arrival time collected in each tile of the detectors were recorded. A coincidence of signals in both VZERO-A and VZERO-C detectors was required to remove contamination from single diffractive and electromagnetic events~\cite{ALICE:2012xs}. The time resolution is better than 1 ns, allowing discrimination of beam--beam collisions from background events produced outside of the interaction region. In the offline analysis, background was further suppressed by the time information recorded in two neutron Zero Degree Calorimeters (ZDCs). A dedicated quartz radiator Cherenkov detector (T0)~\cite{Akindinov:2013tea} provided a measurement of the event time of the collision.

%%In order to reduce the underlying background of jets and improve the resolution of the $\Vzero$ decay vertices, the pileup and bad quality events are rejected by the vertex quality cuts. 
In addition to the trigger selection,  timing and vertex-quality cuts are used to suppress pile-up and bad quality events. The analysis
requires a reconstructed vertex, which is the case for
98.2\% of the events selected by the trigger.
Only the events with reconstructed vertex $v_{z}<10$~cm in beam direction reconstructed with minimum of 2 contributing SPD tracklets are accepted. The total number of events retained in the analysis is $XX M$.

%%%%%%%%%%%%%%%%%%%%%%%%%%%%%%%%%%%%%%%%%%%%%%%%%%%%%%%%%%%%%%%%%%%%%%%%%%%%%%%

\subsection{Event activity estimators}

The selected event sample was divided into three classes (0--10\%, 10--40\%, 40--100\% of the total), based on cuts on two estimators of the event activity:
\begin{itemize}
\item V0A : the amplitude measured by the VZERO hodoscopes on the A-side (the Pb-going side in the p–Pb event sample), $2.8 < \hlab < 5.1$;
\item ZNA : the energy deposited in the neutron calorimeter on the A-side (the Pb-going side in the p–Pb event sample).
\end{itemize}
These event estimators are discussed in detail in \cite{Adam:2014qja}.

\ask{Text below and table \ref{tab:multclasses} needed? - based on the p-Pb ID spectra paper }

The corresponding fractions of the data sample together with the mean charge-particle multiplicity densities ~($\avg{\dNdeta}$) within $| \hlab |<0.5$ in each V0A class are summarized in Tab.~\ref{tab:multclasses}.  
These are obtained using the method presented in~\cite{ALICE:2012xs} and are corrected for acceptance and
tracking efficiency as well as for contamination by secondary
particles. Contrary to our earlier measurement of $\avg{\dNdeta}$~\cite{ALICE:2012xs}, the values
in Tab.~\ref{tab:multclasses} are not corrected for trigger and
vertex-reconstruction efficiency, which is of the order of 2\% for NSD
events~\cite{ALICE:2012xs}. The same holds true for the \pt\
distributions, which are presented in the next section.

\begin{table}[t] 
  \centering
  \begin{tabular*}{\linewidth}{@{\extracolsep{\fill}}ccc}
    \hline
    &&\\[-0.7em]
     Event & V0A range & $\avg{\dNdeta}$\\
     class & \footnotesize{(arb. unit)} & \footnotesize{$|\hlab|<0.5$}\\[0.3em]
    \hline
    &&\\[-0.7em]
    0--10\%   & $>$ 187  & 40.6 $\pm$ 0.9 \\[0.3em]
    10--40\%  & 89-187   & 25.6 $\pm$ 0.6 \\[0.3em]
    40--100\% & $<$ 89   & 10.1 $\pm$ 0.2 \\[0.3em]
    \hline
  \end{tabular*}
  \caption{Definition of the event classes as fractions of the analyzed event sample and their corresponding $\avg{\dNdeta}$ within $|\hlab|<0.5$ (systematic uncertainties only, statistical uncertainties are negligible). \ask{Do we need this table? Also for ZNA?} }
  \label{tab:multclasses}
\end{table}
