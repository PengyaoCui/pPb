%!TEX root = ../AliPubV0JetspPb.tex

\section{Introduction}
%%\ask{Intro taken from the ID spectra mult dependence in p-Pb. Needs adjustments for the purpose of this paper.}

High-energy heavy ion collisions provide a unique opportunity to study
properties of hot and dense QCD medium composed of deconfined partons -
the quark-gluon plasma (QGP). The QGP is predicted by the lattice QCD
calculations \cite{Satz:2000bn,Bass:1998vz,Shuryak:1984nq,Cleymans:1985wb}. The cross-over transition from
hadronic matter to the QGP matter at zero
baryochemical potential is expected to take place once the
temperature of the matter $T_{c}$ reaches values of about 150 MeV and/or
energy density $\epsilon_{c}$ of about 0.5 GeV/fm$^3$ \cite{Borsanyi:2010cj,Bhattacharya:2014ara}. 
The measurements indicate that the most violent collisions of lead ions at the LHC at
\sqrtsnn{2.76}\ create conditions well above the critical temperature at approximately zero baryochemical potential.
The bulk matter created in those collisions can be
quantitatively described in terms of hydrodynamic and statistical
models. The initial hot and dense partonic matter rapidly expands and
cools down, ultimately undergoing a transition to a hadron gas
phase~\cite{Muller:2006ee}. During the expansion phase, collective hydrodynamic flow develops from
the initially generated pressure gradients in the strongly interacting
system. This results in a characteristic dependence of the shape of
the transverse momentum (\pt) distribution on the particle mass that
can be described using a common kinetic freeze-out temperature parameter \Tfo\
and a collective average expansion velocity
\avbT~\cite{Schnedermann:1993ws}.

The interpretation of heavy-ion results depends on the understanding of results from smaller collision
systems such as proton-proton (\pp) or proton-nucleus (pA). Proton-nucleus collisions are intermediate between
proton-proton and nucleus-nucleus collisions in terms of
system size and number of produced particles. Comparing particle
production in pp, pA, and AA reactions has frequently been used to
separate initial state effects, linked to the use of nuclear
beams or targets, from final state effects, associated to the presence of hot and
dense matter. At the LHC, however, the pseudorapidity density of final state
particles in pA collisions reaches values which can become
comparable to semi-peripheral Au--Au ($\sim$60\% most central) and Cu--Cu ($\sim$30\% most central) collisions at top RHIC energy~\cite{Alver:2010ck}.
Indeed, the measurements at the LHC in high-multiplicity pp and \pPb\ collisions have revealed unexpectedly strong long-range correlations of produced particles \cite{Khachatryan:2010gv,CMS:2012qk,Abelev:2012ola,Aad:2012gla,Aad:2013fja,Chatrchyan:2013nka} falsyfying the assumption that final state dense matter effects can be neglected in pA.

Various mechanisms have been proposed to explain the origin of this collective particle production. 
Both a Color Glass Condensate (CGC)
description~\cite{Dusling:2013oia}, based on initial state nonlinear
gluon interactions, as well as a model based on hydrodynamic
flow~\cite{Bozek:2012gr,Qin:2013bha}, assuming strong interactions
between final state partons or hadrons, can give a satisfactory
description of the \pPb\ correlation data. However, the modeling of
small systems such as \pPb\ is complicated because uncertainties
related to initial state geometrical fluctuations play a large role
and because viscous corrections may be too large for hydrodynamics to
be a reliable framework~\cite{Bzdak:2013zma}.

Results on identified particle production in \pPb\ collisions at a nucleon-nucleon center-of-mass energy \sqrtsnn{5.02} at the LHC \cite{Abelev:2013haa} have shown qualitatively similar effects as in AA collisions \cite{Abelev:2013xaa,ABELEV:2013wsa}. In particular the ratio of baryon and meson transverse momentum (\pt) spectra shows a pronounced maximum at intermediate \pt. The shape of the ratio has been discussed in terms of an interplay between the radial expansion of the system and produced particles in a common velovity field (collective flow)~\cite{Schnedermann:1993ws}, soft-hard parton recombination \cite{Fries:2003vb} and hard parton (jet) hadronization at high \pT. Concurently the measurements of jets at mid-rapidty originating from fragmentation of highly-virtual partons produced in hard scatterings within the pA collisions \cite{Adam:2015hoa,Adam:2015xea} have revealed that the final state nuclear effects such as shadowing and gluon saturation (CGC) \cite{McLerran:2001sr,Salgado:2011wc}, or multiple scatterings and hadronic re-interactions in the initial and final state \cite{Krzywicki:1979gv,Accardi:2007in} are not significant. Neither a suppression related to the creation of the QGP in AA collisions was observed \cite{Aad:2010bu,Chatrchyan:2012nia,Aad:2012vca,Abelev:2013kqa,Aad:2014bxa}.

In this letter we report on the measurement of \lda\ and \alda and \ks where the production of particles is studied separately within the region associated to a hard scattering and the remainder of the event (the so called ``underlying event''). The hard scatterings are identified by selecting an energetic jet ($\ptjet > 10 or 20 \gevc$) reconstructed with the anti-\kt\ algorithm with the resolution parameter $R$ of 0.2, 0.3 and 0.4. Moreover, the particle \pt\ spectra and their ratios are reported as a function of their distance to the jet axis and for a selection of the event multiplicity classes of the \pPb\ collisions.


