\subsection{Data sample}
\label{sec:EveSel}

The data used for this analysis was taken with ALICE detector~\cite{Abelev:2014ffa} during the LHC p--Pb run at $\sNN=5.02$~TeV in the beginning of $2013$.
Because of the $2$-in-$1$ magnet design of the LHC~\cite{Evans:2008zzb}, the energy of the two beams are not independent and their ratio is fixed to equal to the ratio of the charge/mass ratios of each beam.
Comparing to the laboratory frame, the nucleon-nucleon center-of-mass system, therefore, was shifting with a rapidity of $\Delta\yNN=0.465$ in the direction of the proton beam.
The used data was collected for the beam configuration, in which the Pb beam circulated in the ``counter-clockwise'' direction, corresponding to travel from ALICE C to A side or positive rapidity.

The ALICE apparatus is detailedly described in~\cite{Aamodt:2008zz}.
The minimum-bias trigger signal was provided by the VZERO~\cite{Abbas:2013taa} counters, the VZERO-A in Pb beam direction and the VZERO-C in proton beam direction covering the pseudo-rapidity $2.8<\hlab<5.1$ and $-3.7<\hlab<-1.7$, respectively.
A coincidence of signals in both VZERO-A and VZERO-C detectors was required to remove contamination from single diffractive and electromagnetic events~\cite{ALICE:2012xs}.
The resolution of the arrival time is better than $1$~ns, allowing discrimination of beam--beam collisions from background events produced outside of the interaction region.
In the offline analysis, background was further suppressed by the time information recorded in two neutron Zero Degree Calorimeters (ZDCs), which located at $+112.5$~m (ZNA) and $-112.5$~m (ZNC) from the interaction point.
A dedicated quartz radiator Cherenkov detector (T0)~\cite{Akindinov:2013tea} provided a measurement of the event time of the collision.

The ALICE central-barrel tracking detectors that have full azimuthal coverage in the pseudo-rapidity interval $|\hlab|<0.9$.
Tracking and particle identification in this Letter are mainly performed using the information provided by the Inner Tracking System (ITS)~\cite{Aamodt:2010aa} and the Time Projection Chamber (TPC)~\cite{Alme:2010ke} detectors.
In addition to the trigger selection, the events are further selected by requiring with reconstructed vertex $v_{z}<10$~cm along beam axis and that vertices from SPD (the two innermost Silicon Pixel Detector (SPD) layers of ITS) tracklets and global tracks are compatible.
The total number of events retained in the analysis is 96~M.