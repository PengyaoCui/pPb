\section{Binomial Errors}
\label{sec:a02BinoErr}

In general, efficiency is:
\begin{equation}\label{eq:a02EffiGen}
\varepsilon=r/g,
\end{equation}
where,
$r$ is number of reconstructed particles and
$g$ is number of generated particles.
Define:
\begin{equation}\label{eq:a02DefD}
d=g-r,
\end{equation}
eq.~(\ref{eq:a02EffiGen}) becomes:
\begin{equation}
\varepsilon=r/(r+d).
\end{equation}

Since $r$ and $d$ are the independent variables,
the uncertainty of $\varepsilon$ is given by:
\begin{equation}\label{eq:a02BinoErr}
\sigma_{\varepsilon}^{2}=
(\frac{\partial\varepsilon}{\partial r}\sigma_{r})^{2}+
(\frac{\partial\varepsilon}{\partial d}\sigma_{d})^{2}=
\varepsilon(1-\varepsilon)/g,
\end{equation}
where, $\sigma_{r}^{2}=r$, $\sigma_{d}^{2}=d$.
Eq.~(\ref{eq:a02BinoErr}) gives the binomial error for efficiency
and it also concludes the statistic uncertainty calculation for the formulae
which have the same format as eq.~(\ref{eq:a02EffiGen}).

By considering the branching ratio ($r_{\rm b}$) in the efficiency,
the parameter $d$ in eq.~(\ref{eq:a02DefD}) has to be redefined as:
\begin{equation}
d=b\cdot g-r,
\end{equation}
then, the binomial error of $\varepsilon$ becomes:
\begin{equation}\label{eq:a02BinoErrBR}
\sigma_{\varepsilon}^{2}=
\frac{r_{\rm b}-\varepsilon}{r_{\rm b}\cdot g}\varepsilon.
\end{equation}
