\section{Introduction}
%%%%%%%%%%%%%%%%%%%%%%%%%%%%%%%%%%%%%%%%%%%%%%%%%%%%%%%%%%%%%%%%%%%%%%%%%%%%%%%

This analysis is aimed to shed light on the enhanced baryon to meson ratio
in p--A collisions as compared to pp collisions.
The $p_{\rm T}$ differential production yields of $\Lambda$ baryons
and ${\rm K}_{\rm S}^{0}$ mesons are measured separately in the region of
hard scattering tagged with charged particle jet and in the region dominated
by the soft particle production: the underlying
event (not tagged by the hard scattering).

The $\Lambda$ and ${\rm K}_{\rm S}^{0}$ are reconstructed via invariant mass
analysis.
The jets are reconstructed with the so called hybrid track selection with
anti-$k_{\rm T}$ jet finder with $R=0.2, 0.3$ and $0.4$.
The studies are performed as a function of event activity (event class)
according to V0M and ZNA.
In each case we find that the peak observed in the
inclusive $\Lambda$-to-${\rm K}_{\rm S}^{0}$ ratio is characteristic to the
particles originating from the underlying event,
whereas the $\Lambda$-to-${\rm K}_{\rm S}^{0}$ ratio associated to the
charged particle jets (with $p_{\rm T}>10$ and $20~{\rm GeV}/c$) is consistent
with Pythia simulation of hard scatterings in p-p collisions.

Documents:
\begin{itemize}
\item presentation of preview:
\href{https://indico.cern.ch/event/311402/session/11/contribution/67/material/slides/0.pdf}
     {https://indico.cern.ch/event/311402/session/11/contribution/67/material/slides/0.pdf}.
\end{itemize}
