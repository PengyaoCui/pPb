%!TEX root = ../AliPubV0JetspPb.tex

To sched light on the origin of the so-called ``baryon enhancement'' observed in \pPb\ collisions at \sqrtsnn{5.02} at the LHC the production of \lda\ baryons and \ks\ mesons was measured separately in hard scattering region and the underlying event.
The hard scatterings are selected on an event-by-event basis by jets reconstructed using charged particles with \akT\ jet finder. 
The production of strange particles is reported as a function of their transverse momentum \pt\ and the angular distance $R$ from the jet axis for jets with $\ptch > 10$~\gevc\ and $\ptch > 20$~\gevc.

%%The ratio of inclusive differential yields of \lda\ and \ks\ at intermediate transverse momentum (\pt) is much larger in the systems such as \PbPb\ and \pPb\ collisions as compared to \pp\ collisions. The increased ratio in \PbPb\ has been attributed to collective effects in those collisions. Recent studies have revealed qualitatively similar effects in high-multiplicity \pPb\ collisions.

For small $R$ ($R < 1.0$) the $(\lda+\alda)/\ks$ ratio associated to jets is found consistent with the expectation of jets fragmenting in vacuum given by \pythia\ event generator. 
In contrast, this ratio for large $R$ corresponding to the underlying event shows a maximum similar to that of inclusive production in \pPb\ collisions at the intermediate \pt\ of 2-5 \gevc. 

\ask {Jana: A suggestion to the abstract: maybe we should add a sentence on the motivation of our measurement at the beginning of the abstract. What do you think?}

%Moreover the yields in jets do not change with the event multiplicity, while the large baryon/meson ratio evolves and it is larger in events with the highest multiplicity as compared to minimum bias collisions.


